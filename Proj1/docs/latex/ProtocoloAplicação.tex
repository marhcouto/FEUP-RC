\section{Protocolo de Aplicação}

O Protocolo de Aplicação implementado tem como objetivos:
\begin{enumerate}
    \item Envio, receção e construção dos pacotes de controlo de início e fim de transmissão
    \item Leitura/Escrita de dados de/para uma porta série utilizando o protocolo de ligação lógica
    \item Construção/Interpretação de pacotes de dados com \textit{headers} válidos
    \item Divisão de um ficheiro em fragmentos e posterior envio (transmissor)
    \item Construção de um ficheiro através de dados recebidos (recetor)
\end{enumerate}


\subsection{Pacotes de Controlo}

Os pacotes de controlo de transmissão são construídos pelo transmissor para denotar o inicio e fim de transmissão, bem como o envio de informação sobre os dados a serem enviados, tal como o tamanho do ficheiro e o seu nome. 

\begin{lstlisting}[language=C, caption=Função que constrói o pacote de controlo (transmissor)]
uint8_t* build_control_packet(size_t* control_packet_size, char* file_name) 
\end{lstlisting}

\begin{lstlisting}[language=C, caption=Função que interpreta o pacote de controlo recebido (recetor)]
void parse_control_packet(uint8_t* packet) 
\end{lstlisting}

O tamanho do ficheiro, enviado no inicio da ligação no pacote de controlo, é usado para verificar, no fim da ligação, se o número de bytes recebidos corresponde ao esperado.

\begin{lstlisting}[language=C, caption=Verificação do tamanho do ficheiro recebido (receiver)]
    if (bytes_received != app_data.transfer_file_size) {
        fprintf(stderr, "File size does not correspond to the expected\n");
    }
\end{lstlisting}

\subsection{Leitura e Escrita de dados}

A leitura e escrita de dados são ambos efetuados em loops nas funções principais do emissor e recetor, \textbf{transmitter} e \textbf{receiver} respetivamente.

\begin{lstlisting}[language=C, caption=Envio de pacotes de dados]
while (bytes_read < app_data.transfer_file_size) {
    packet_no++;
    bytes_written = read_data_packet(packet_no, data_packet, oldFileDescriptor, bytes_read);
    bytes_read += bytes_written - 4;
    if ((llwrite(app_data.file_descriptor, data_packet, bytes_written)) == -1) exit(1);
    printf("Transmitter - packet %d sent\n", packet_no);
}
\end{lstlisting}


\begin{lstlisting}[language=C, caption=Leitura dos pacotes de dados]
    while (true) {
        // READ
        if (llread(app_data.file_descriptor, data_packet, MAX_PACKET_SIZE) == -1) exit(1);
        if (data_packet[0] == 0x01) {
            bytes_received += write_data_packet(data_packet, newFileDescriptor, &packet_no);
            printf("Receiver - packet %d received\n", (int) data_packet[1]);
        } else if (data_packet[0] == 0x02) {
            parse_control_packet(data_packet);
            if (file_exists(app_data.transfer_file_dir, app_data.transfer_file_name)) {
                fprintf(stderr, "Error in application - receiver: file already exists\n");
                exit(1);
            }
            newFileDescriptor = open_transfer_file();
            fileOpen = true;

        } else if (data_packet[0] == 0x03) {
            // CLOSE
            if ((llclose(app_data.file_descriptor)) != 0) exit(1);
            printf("Receiver - connection cut\n");
            break;
        } else {
            fprintf(stderr, "Error in application - receiver: C byte value invalid\n");
            exit(1);
        }
    }
\end{lstlisting}

Nos pacotes de dados possuem numeração, que permitem avisar qualquer inconsitência nos pacotes recebidos. Esta verificação é feita na função \texttt{read\_data\_packet}. \textbf{Nota:} esta funcionalidade foi implementada posteriormente à apresentação.
    
\subsection{Leitura do ficheiro e construção dos pacotes}

A leitura do ficheiro do \textbf{emissor} e subsequente construção do pacote de dados é feita incrementalmente, sendo a seguinte função chamada no loop de escrita referido anteriormente.

\begin{lstlisting}[language=C, caption=Função que lê o ficheiro e constrói o pacote de dados]
    size_t read_data_packet(int packetNo, uint8_t* data_packet, int fd, int bytes_read)
\end{lstlisting}

\subsection{Interpretação dos pacotes e escrita do ficheiro}

O \textbf{recetor} executa a interpretação e verificação dos dados recebidos e a escrita dos mesmos num ficheiro numa mesma função, chamada iterativamente no loop acima descrito.

\begin{lstlisting}[language=C, caption=Função que interpreta o pacote de dados recebido e escreve no ficheiro]
    size_t write_data_packet(uint8_t* data_packet, int fd, int* packet_no)
\end{lstlisting}
