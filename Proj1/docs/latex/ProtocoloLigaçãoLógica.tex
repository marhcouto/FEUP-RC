\section{Protocolo de Ligação Lógica}

Os objetivos principais da camada de ligação de dados são:

\begin{enumerate}
    \item Redução e controlo dos erros de transmissão
    \item Regulação do fluxo de dados
    \item Disponibilização de uma interface para comunicação através da porta série à camada de aplicação
\end{enumerate}



\subsection{Leitura da Porta Série}

Para leitura da porta série adaptada da máquina de estados fornecida, para suportar frames de dados, e tem como intuito ler as tramas recebidas na porta e verificar a sua validade. 

A máquina de estados lê dados da porta série byte a byte. Depois da receção do BCC1, se não for detetada uma FLAG, sabemos que estamos perante uma frame de informação. Os dados são colocados num buffer e a leitura acaba quando é encontrada uma FLAG. O buffer contém os dados até à posição N - 1 pois o dado de posição N é o BCC2. Em seguida apresenta-se a sua implementação em pseudocódigo.  

\begin{lstlisting}[caption=Pseudocódigo da StateMachine]
    buffer <- Empty
    While state != STOP:
        byte_rcvd <- read_from_serial_port
        if (state == START) then:
            if byte_rcvd == FLAG then:
                state <- FLAG_RCVD
        if (state == FLAG_RCVD):
            if (isA(byte_rcvd)) then:
                state <- A_RCVD
        if (state == A_RCVD) then:
            if (isC(byte_rcvd)) then:
                state <- A_RCVD
            ElseIf (byte_rcvd == FLAG) then:
                state = FLAG_RCVD
            Else:
                state = START
        if (state == C_RCVD) then:
            if (validBCC1()) then:
                state = BCC_OK
            ElseIf (byte_rcvd == FLAG) then:
                state = FLAG_RCVD
            Else:
                state = START
        if (state == BCC_OK) then:
            if (byte_rcvd == FLAG_RCVD) then:
                state <- STOP
            Else:
                Insert(buffer, byte_rcvd)
                state = DATA
        if (state == DATA) then:
            if (byte_rcvd == FLAG) then:
                state = STOP
            Else:
                Insert(buffer, byte_rcvd)
\end{lstlisting}

Quando a verificação do BCC1 falha não se envia qualquer resposta uma vez que é impossível determinar o tipo de frame recebido.

Cada frame é lido num loop para o caso de haver erros na parte de informação e ser necessário receber outro frame. Se o BCC2 for válido e o frame corresponder ao número esperado um RR é enviado senão é enviado REJ e aguarda-se pelo novo frame.

\begin{lstlisting}[language=C, caption=Código em C do loop responsével pela leitura da porta série, em llread]
while(!valid_frame) {
    new_frame = read_frame(fd, link_layer_data.frame, MAX_FRAME_SIZE);
    frame_size = destuff(link_layer_data.frame, new_frame.packet_size, aux_buffer, MAX_PACKET_SIZE + 1);

    if (frame_size == 0) {
        fprintf(stderr, "Error in data_link - llread: destuffing packet unsuccessfull\n");
        return -1;
    }
    uint8_t bcc2 = aux_buffer[--frame_size];

    // Error
    if (new_frame.frame_type != expected_package || !validBCC2(bcc2, aux_buffer, frame_size)) {
        if (assemble_supervision_frame(expected_package ? REJ_1 : REJ_0, RECEIVER, response_frame) == -1) return -1;
    } else {
        memcpy(buffer, aux_buffer, frame_size);
        if (assemble_supervision_frame(expected_package ? RR_0 : RR_1, RECEIVER, response_frame) == -1) return -1;
        valid_frame = true;
    }
    if (write(fd, response_frame, SUPV_FRAME_SIZE) != 5) {
        fprintf(stderr, "Error in data_link - llread: did not write response frame correctly");
        return -1;
    }
}
\end{lstlisting}

\subsection{Escrita na Porta Série}

Quando um buffer é escrito, o utilizador da função deve especificar uma resposta esperada pelo outro lado da porta série. Quando o buffer é enviado, é acionado um alarme. Se o alarme for chamado uma flag é ativada e retira-se um valor ao número de tentativas de leitura. Se o número de tentativas for esgotado, assume-se que a trama não foi enviada.
\begin{lstlisting}[language=C, caption=Código em C da função responsável pela escrita na porta série]
/* Writes a buffer to the serial port and awaits a certain response
    fd -> Serial port file desciptor
    buffer -> buffer to write
    length -> length of the buffer to be written
    expected_response -> expected response sent by the other end of the serial port
*/
int write_buffer(int fd, uint8_t* buffer, size_t length, FrameType expected_response) 
\end{lstlisting}

\subsection{Interface para uso na camada de aplicação}

O programa foi desenhado de modo a manter independência entre a camada de aplicação e o protocolo de ligação de dados, de modo que este último é apenas visível à camada superior como 4 pontos de entrada.

\subsubsection{Estabelecimento da ligação}

Para o estabelecimento de ligação através de uma porta série, a função disponibilizada \textbf{llopen}. 

\begin{itemize}
    \item Emissor - configura-se a porta série, envia-se uma trama \textbf{SET} (pelo emissor) e aguarda-se uma trama \textbf{UA} que será enviada pelo recetor
    \item Recetor - configura-se a porta série, aguarda-se a receção de uma trama \textbf{SET}; quando esta é recebida corretamente, envia-se uma trama \textbf{UA} de modo a marcar o estabelecimento de ligação
\end{itemize}

\begin{lstlisting}[language=C, caption=llopen header]
int llopen(char* port, ConnectionType role)
\end{lstlisting}

\subsubsection{Término da conexão}

Para o fecho da ligação, o protocolo de ligação de dados fornece a função \textbf{llclose}.

\begin{itemize}
    \item Emissor - envia-se uma trama \textbf{DISC}, aguarda-se por uma trama \textbf{DISC} e finalmente envia-se uma trama \textbf{UA}.
    \item Recetor - aguarda-se uma trama \textbf{DISC}, envia-se uma trama \textbf{DISC} e finalmente aguarda-se uma trama \textbf{UA}.
\end{itemize}


\begin{lstlisting}[language=C, caption=llclose header]
int llclose(int fd)
\end{lstlisting}


\subsubsection{Envio de dados}

Para o envio de dados (pela parte do emissor), existe a função \textbf{llwrite}.

Esta função apenas invoca a função \texttt{write\_buffer} descrita acima, além de controlar o número de sequência da trama.

\subsubsection{Receção de dados}

Para a receção de dados (pela parte do recetor), é providenciada a função \textbf{llread}.

Esta função invoca a \texttt{state\_machine} e controla o processo de deteção de erros descritos acima.