\section{Anexo 1 - Código da aplicação FTP}

\paragraph{Modules}

\begin{itemize}
    \item \textbf{ftp} - ligação ao servidor e transferência do ficheiro
    \item \textbf{main} - função inicial
    \item \textbf{utils} - funções genéricas que auxiliam o programa
    \item \textbf{url\_path\_parser} - interpretação dos argumentos (URL fornecido)
    \item \textbf{macros} - constantes
\end{itemize}

\subsection{main}

\begin{lstlisting}[language=C, caption=main.c]
#include <stdio.h>
#include "ftp.h"
#include "url_path_parser.h"

URLPathData resource_location;

void print_usage() {
    printf("Usage: download resource_location\n");
}

int main(int argc, char* argv[]) {
    setbuf(stdout, NULL);
    if (argc == 2) {
        if (create_url_data(argv[1], &resource_location) != 0) {
            return -1;
        }
        return ftp_download(&resource_location);
    } else {    
        print_usage();
        return -1;
    }
    return 0;
}
\end{lstlisting}


\subsection{ftp}
\begin{lstlisting}[language=C, caption=ftp.c]
#include <stdio.h>
#include <unistd.h>
#include <sys/types.h>
#include <sys/socket.h>
#include <arpa/inet.h>
#include <stdlib.h>
#include <string.h>
#include <errno.h>
#include <sys/select.h>
#include "macros.h"
#include "ftp.h"
#include "utils.h"

static URLPathData* resource_location;
static int server_socket;
static int data_socket;
static FILE* server_socket_file;
static char* buffer;
unsigned long file_size = 0;
int ftp_control_connect(URLPathData* path_data);
int ftp_data_connect(DataConData* data_con);
void ftp_close();
int ftp_dump_data(int fd);
int ftp_read_resp_code();
int ftp_login();
int ftp_retrieve_passive_mode(DataConData* data_con_data);
int ftp_change_dir_to_res(URLPathData* url_data);


// Main function to execute download
int ftp_download(URLPathData* path_data) {
    int fd;
    DataConData con_data;

    if (ftp_control_connect(path_data) == -1) {
        return -1;
    }
    if (ftp_login() == -1){
        return -1;
    }
    if (ftp_retrieve_passive_mode(&con_data) == -1) {
        return -1;
    }
    if (ftp_data_connect(&con_data) == -1) {
        return -1;
    }
    if (ftp_change_dir_to_res(path_data) == -1) {
        return -1;
    }
    if((fd = f_retrieve_fd(path_data)) == -1) {
        return -1;
    }
    if (ftp_dump_data(fd) == -1) {
        return -1;
    }
    ftp_close();
    return 0;
}

// Connect to server
int ftp_control_connect(URLPathData* path_data) {
    resource_location = path_data;
    buffer = calloc(MAX_CTRL_SIZE, sizeof(char));
    struct sockaddr_in server_addr = {
        .sin_family = AF_INET,
        .sin_addr.s_addr = inet_addr(inet_ntoa(*((struct in_addr *) resource_location->sv_addr))),
        .sin_port = htons(path_data->port),
    };
    if ((server_socket = socket(AF_INET, SOCK_STREAM, 0)) == -1) {
        perror("ftp_control_connect_socket");
        return -1;
    } 
    if (connect(server_socket, (struct sockaddr *) &server_addr, sizeof(server_addr)) == -1) {
        perror("ftp_control_connect_connect");
        return -1;
    }
    if ((server_socket_file = fdopen(server_socket, "r")) == NULL) {
        perror("ftp_control_connect_fdopen");
        return -1;
    }
    if (ftp_read_resp_code() != 220) {
        fprintf(stderr, "Error: Server responded with unexpected code!\n");
        return -1;
    }
    return 0;
}

// Establishes data connection
int ftp_data_connect(DataConData* data_con) {
    struct sockaddr_in server_addr = {
        .sin_family = AF_INET,
        .sin_addr.s_addr = inet_addr(data_con->ip),
        .sin_port = htons(data_con->port),
    };
    if ((data_socket = socket(AF_INET, SOCK_STREAM | SOCK_NONBLOCK, 0)) == -1) {
        perror("ftp_data_connect_socket");
        return -1;
    } 
    if ((connect(data_socket, (struct sockaddr *) &server_addr, sizeof(server_addr)) == -1) && (errno != EINPROGRESS)) {
        perror("ftp_data_connect_connect");
        return -1;
    }
    return 0;
}

// Close connection
void ftp_close() {
    close(server_socket);
    close(data_socket);
}

// Download
int ftp_dump_data(int fd) {
    struct timeval tv;
    fd_set socket_set;
    uint8_t byte_stream[1024];
    ssize_t read_size;
    FD_ZERO(&socket_set);
    FD_SET(data_socket, &socket_set);
    tv.tv_sec = 5;
    tv.tv_usec = 0;

    int select_retval = select(data_socket + 1, &socket_set, NULL, NULL, &tv);
    if (select_retval == -1) {
        perror("select");
        return -1;
    } else if (select_retval == 0) {
        fprintf(stderr, "Error: Couldn't establish connection with data link.\n");
        return -1;
    }
    printf("Starting download...\n");

    unsigned long downloaded_bytes = 0;

    while((read_size = read(data_socket, byte_stream, 1024)) != 0) {
        write(fd, byte_stream, read_size);
        downloaded_bytes += read_size;
        print_progress(downloaded_bytes, file_size);
    }
    printf("\n");
    write(fd, byte_stream, read_size);
    if (ftp_read_resp_code() != 226) {
        fprintf(stderr, "Error: Unexpected response after end of transfer!\n");
        return -1;
    }
    return 0;
}

// Gets server resonse
int ftp_read_resp_code() {
    do {
        memset(buffer, 0, MAX_CTRL_SIZE);
        fgets(buffer, MAX_CTRL_SIZE, server_socket_file);
    } while (buffer[0] < '1' || buffer[0] > '5' || buffer[3] != ' ');
    printf("%s", buffer);
    buffer[3] = '\0';
    int response = atoi(buffer);
    if (response == 150) file_size = read_file_size(buffer + 4); // To print the progress percentage
    return response;
}

// Login
int ftp_login() {

    // User
    if (resource_location->username != NULL) {
        strcpy(buffer, "user ");
        strcat(buffer, resource_location->username);
    } else {
        strcpy(buffer, "user anonymous");
    }
    strcat(buffer, "\r\n");
    write(server_socket, buffer, strlen(buffer));
    printf("%s", buffer);
    if (ftp_read_resp_code() != 331) {
        fprintf(stderr, "Error: Unexpected response after username was sent!\n");
        return -1;
    }

    // Password
    if (resource_location->password != NULL) {
        strcpy(buffer, "pass ");
        strcat(buffer, resource_location->password);
    } else {
        strcpy(buffer, "pass teste");
    }
    strcat(buffer, "\r\n");
    write(server_socket, buffer, strlen(buffer));
    printf("%s", buffer);
    if (ftp_read_resp_code() != 230) {
        fprintf(stderr, "Error: Unexpected response after password was sent!\n");
        return -1;
    }
    return 0;
}

// Request port for data connection (passive mode FTP)
int ftp_retrieve_passive_mode(DataConData* data_con_data) {
    unsigned ip[4];
    unsigned msb_port;
    unsigned lsb_port;
    strcpy(buffer, "pasv\r\n");
    write(server_socket, buffer, strlen(buffer));
    fscanf(server_socket_file, "227 Entering Passive Mode (%u,%u,%u,%u,%u,%u).", 
        &ip[0],
        &ip[1],
        &ip[2],
        &ip[3],
        &msb_port,
        &lsb_port
    );
    snprintf(data_con_data->ip, 100, "%u.%u.%u.%u", ip[0], ip[1], ip[2], ip[3]);
    data_con_data->port = msb_port * 256 + lsb_port;
    return 0;
}

// Changes the current directory to the directory of the file to download
int ftp_change_dir_to_res(URLPathData* url_data) {
    size_t url_size = strlen(url_data->url) + 1;
    char* url_cpy = calloc(url_size, sizeof(char));
    strncpy(url_cpy, url_data->url, url_size);

    char* last_token = strrchr(url_cpy, '/') + 1;
    char* token = strtok(url_cpy, "/");
    while(token != NULL) {
        if (token == last_token) {
            break;
        }
        strcpy(buffer, "CWD ");
        strcat(buffer, token);
        strcat(buffer, "\r\n");
        write(server_socket, buffer, strlen(buffer));
        if (ftp_read_resp_code() != 250) {
            fprintf(stderr, "Error: Invalid path to resource!\n");
            return -1;
        }
        token = strtok(NULL, "/");
    }
    strcpy(buffer, "RETR ");
    strcat(buffer, last_token);
    strcat(buffer, "\r\n");
    write(server_socket, buffer, strlen(buffer));
    if (ftp_read_resp_code() != 150) {
        fprintf(stderr, "Error: Unexpected response to RETR command!\n");
        return -1;
    }
    free(url_cpy);
    return 0;
}
\end{lstlisting}


\begin{lstlisting}[language=C, caption=ftp.h]
#pragma once

#include <stdint.h>
#include "url_path_parser.h"

typedef struct {
    char ip[17];
    uint16_t port;
} DataConData;

int ftp_download(URLPathData* path_data);
\end{lstlisting}

\subsection{url\_path\_parser}

\begin{lstlisting}[language=C, caption=url\_path\_parser.c]
#include <stdio.h>
#include <stdlib.h>
#include <stdbool.h>
#include <string.h>
#include <netdb.h>
#include "url_path_parser.h"

// Prints error message
void print_url_format_error() {
    fprintf(stderr, "Invalid URL format. It should respect the following convention: ftp://<user>:<password>@<host>:<port>/<url-path>\n");
}

// Checks it url is FTP
bool is_ftp(char* url_path) {
    if (strncmp(url_path, "ftp://", 6) == 0) {
        return true;
    }
    return false;
}

// Cheks if path has port
bool has_port(char* url_path) { 
    return (strchr(url_path, ':') != NULL);
}

char* strptrcpy(char* start, char* end) {
    size_t len;
    if (end == NULL) {
        len = strlen(start);
    } else {
        len = (size_t) (end - start);
    }
    char* dest = calloc(len + 1, sizeof(char));
    for(size_t i = 0; i < len && start != end; i++) {
        dest[i] = *start;
        start++;
    }
    dest[len] = '\0';
    return dest; 
}

// Parses the information from the url to a struct
int create_url_data(char* url_path, URLPathData* url_data) {
    char* server_name;
    struct hostent* sv_resesolv;
    url_data->port = 21;

    // Check if ftp
    if (!is_ftp(url_path)) {
        print_url_format_error();
        return -1;
    }

    // User and password
    url_path += strlen("ftp://"); //Removes protocol from url path
    char* at_place = strchr(url_path, '@');
    if (at_place != NULL) {
        char* user_pass_sep = strchr(url_path, ':');
        if (user_pass_sep == NULL || user_pass_sep > at_place) {
            print_url_format_error();
            return -1;
        }
        url_data->username = strptrcpy(url_path, user_pass_sep);
        url_data->password = strptrcpy(user_pass_sep + 1, at_place);
        url_path = at_place + 1;
    }

    char* resource_slash = strchr(url_path, '/');
    if(resource_slash == NULL) {
        print_url_format_error();
        return -1;
    }

    // Port
    if (has_port(url_path)) {
        char* port_tok = strchr(url_path, ':');
        if (port_tok == NULL) {
            print_url_format_error();
            return -1;
        }
        server_name = strptrcpy(url_path, port_tok);
        char* port_str = strptrcpy(port_tok + 1, resource_slash);
        url_data->port = (uint16_t)atoi(port_str);
        free(port_str);
    } else {
        server_name = strptrcpy(url_path, resource_slash);
    }


    // Url
    url_data->url = strptrcpy(resource_slash + 1, NULL);
    if ((sv_resesolv = gethostbyname(server_name)) == NULL) {
        herror("create_url_data");
        return -1;
    }
    
    // Server
    url_data->sv_addr = (char*) calloc(strlen(sv_resesolv->h_addr) + 1, sizeof(char));
    strcpy(url_data->sv_addr, sv_resesolv->h_addr);
    free(server_name);
    return 0;
}

void destroy_url_data(URLPathData data) {
    free(data.username);
    free(data.password);
    free(data.url);
    free(data.sv_addr);
}
\end{lstlisting}



\begin{lstlisting}[language=C, caption=url\_path\_parser.h]
#pragma once

#include <stdint.h>

typedef struct {
    char* username;
    char* password;
    char* url;
    char* sv_addr;
    uint16_t port;
} URLPathData;

int create_url_data(char* url_path, URLPathData* url_data);
char* strptrcpy(char* start, char* end);
void destroy_url_data(URLPathData data);
\end{lstlisting}


\subsection{utils}
\begin{lstlisting}[language=C, caption=utils.c]
#include <stdio.h>
#include <unistd.h>
#include <string.h>
#include <sys/stat.h>
#include <sys/types.h>
#include <stdlib.h>
#include <fcntl.h>
#include "utils.h"



int f_retrieve_fd(URLPathData* url_data) {
    if (url_data == NULL) {
        fprintf(stderr, "Error: Invalid path!\n");
        return -1;
    }
    char* file_name = strrchr(url_data->url, '/') + 1 ?: url_data->url;
    if (access(file_name, F_OK) != -1) {
        fprintf(stderr, "Missing permissions or file already exists\n");
        return -1;
    }
    return creat(file_name, S_IRUSR | S_IWUSR | S_IRGRP | S_IROTH);
}


// Reads file size from server message
unsigned long read_file_size(char* buffer) {
    int i = 0;
    char* file_size_str;
    while(buffer[i] != 40) i++;

    i++;
    file_size_str = buffer + i;
    while(buffer[i] != 32) i++;

    buffer[i] = '\0';

    return atoll(file_size_str);
}

// Prints progress message
void print_progress(unsigned long progress, unsigned long total) {
    if (total == 0) {
        fprintf(stderr, "\rDownload file size not defined\n");
        return;
    }
    double percentage = ((double) progress) / ((double) total) * 100;
    printf("\rDownloaded %lf%%                     ", percentage);
    sleep(1);
}
\end{lstlisting}



\begin{lstlisting}[language=C, caption=utils.h]
#pragma once

#include "url_path_parser.h"

int f_retrieve_fd(URLPathData* url_data);
unsigned long read_file_size(char* buffer);
void print_progress(unsigned long progress, unsigned long total);
\end{lstlisting}


\subsection{macros}
\begin{lstlisting}[language=C, caption=macros.h]
#pragma once

#define MAX_CTRL_SIZE 1024
\end{lstlisting}
